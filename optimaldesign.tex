\documentclass[12pt]{iopart}
\usepackage{graphicx}
\usepackage{iopams}
\begin{document}
\article[Optimal design for spectroscopy]{TECHNICAL DESIGN NOTE}{Optimal design for spectroscopy}
\author{G Imreh and W-Y Cheng}
\address{Institute of Atomic and Molecular Sciences, Academia Sinica, Taiwan}
\ead{\mailto{imrehg@gmail.com}}
\begin{abstract}
Optimal design methodology for spectroscopy and other fitted things.
\end{abstract}

\noindent{\it Keywords\/}:

\pacs{}
\submitto{\MST}

\section{Introduction}



Let's discuss some ideas how to make a better spectroscopy measurement by the power of statistics \cite{Atkinson1992}.

When doing measurements, one wants to get the best result with the smallest effort....




\textbf{Overview/roadplan:}
\begin{enumerate}
\item Overview of optimal design
\item Discrete support design function
\item Extending to continuous support
\item Overall optimization / selective optimization
\item Algorithms
\item Application to spectroscopy
\end{enumerate}

\section{Optimal design}
Short introduction of linear regression using matrix algebra, only to show the notation in this paper. For more detailed explanation check any standard linear algebra text book. Measure $y_i$ at $x_i$ points ($i=1, \ldots, N$)
\begin{equation}
E[Y] = F \beta 
\end{equation}
where $Y$ is the $N \times 1$ vector of responses, $\beta$ is a $p \times 1$ vector of unknown parameters, and $F$ is the $N \times p$ extended design matrix. The $i$th row of $F$ is $f^T(x_i)$, a known function of the $m$ explanatory variables.

Least squares regression is then:
\begin{equation}
\hat \beta = (F^T F)^{-1} F^T y
\end{equation}
The covariance matrix of the least squares estimates is
\begin{equation}
\mathrm{var} \hat \beta = \sigma^2 (F^T F)^{-1}
\end{equation}


\subsection{Obtimal design}
Over $\xi$ is a continuous measure over $\mathcal{X}$. If $\xi$ has $n$ distinct points, then
\begin{equation}
\xi = \left\{ 
  \begin{array}{l l l l}
    x_1 & x_2 & \ldots & x_n\\
    w_1 & w_2 & \ldots & w_n\\
  \end{array} \right \}
\end{equation}
where $w_i$ is the weight of point $x_i$, $\int_{\mathcal{X}}\xi(dx) = 1$ and $0 \leq w_i \leq 1$ for every $i$.

Exact design is for realizable integers for a given $N$.
\begin{equation}
\xi_N = \left\{ 
  \begin{array}{l l l l}
    x_1 & x_2 & \ldots & x_n\\
    r_1/N & r_2/N & \ldots & r_n/N\\
  \end{array} \right \}
\end{equation}

The N-trial design:
\begin{equation}
F^T F = \sum_{i=1}^N f(x_i) f(x_i)^T
\end{equation}.
The information matrix is
\begin{equation}
M(\xi) = \int_{\mathcal{X}} f(x)f^T(x) \xi(dx) = \sum_{i=1}^n f(x_i)f^T(x_i)w_i
\end{equation}
summing over the $n$ design points.
For the exact design we have to scale as
\begin{equation}
M(\xi_N) = \frac{F^T F}{N}
\end{equation}
The variance of the response is given by
\begin{equation}
\mathrm{var}\{\hat y(x)\} = \sigma^2 f^T(x) M^{-1}(\xi) f(x)
\end{equation}
from where the standardized variance of the predicted response is
\begin{equation}
d(x, \xi) = f^T(x) M^{-1}(\xi) f(x)
\end{equation}
that is function of both $x$ where the prediction is made and the design $\xi$.

Talk about the effect of singular matrices.

There is no standard way to caulculate the optimal $\xi$ in general. There are methods developed (see citation, ...). One subset of methods is sequential design: adding points one-by-one, or removing points one-by-one. One practical method is outlined here:

\begin{enumerate}
\item Nonsingular start with $n$ points
\item Calculate argmax $d(x, \xi_n)$ for $x \in \mathcal{X}$,
\item Add that point to the bunch as $n+1$
\item Repeat until algorithm converges
\item Can throw away starting points
\item Should have an approximate measure of the optimal $\xi^*$
\end{enumerate}
Works for any differentiable functions, even if only differentiable numerically.

\subsection{Restricted optimal design}

\begin{equation}
d_s(x, \xi) = f^T(x) M^{-1}(\xi)f(x) - f_2^T(x) M_{22}^{-1}(\xi)f_2(x)
\label{eq:dsvar}
\end{equation}

\subsection{Sub-obtimal design for robustness}

Surprisingly robust, the $\hat \beta$ has to be very far from the true to lose power... Do simulation!

\subsection{Finite sample size}

$D_\mathrm{eff}$ gives approximate measure of the save effort.


\subsection{Design for restricted parameter space}

Doing the same method for restricting $\mathcal{X}$ is just as possible. 

\subsection{Heterosadicity solved by GLM}
\subsection{Model validation}

\subsection{Imprecision of $\hat \beta$}

How to calculate the predicted effect of imprecise parameters?

\subsection(Functions of variables)

\begin{equation}
\mathrm{var}\left\{g(\hat \theta)\right\} = \mathrm{var}(c^T \hat \beta) = c^T M^{-1}(\xi) c
\end{equation}
where $c$ is a $p \times 1$ vector of known constans. Generally, it can be defined from the Tailor expansion:
\begin{equation}
c_i(\theta) = \frac{\partial g(\theta)}{\partial \theta_i}
\end{equation}

Have to be careful, though, because the model can be singular. For linear models this means maybe only linear combinations of parameters can be estimated, for nonlinear models it is possible that the parameter won't be observable. Care needed!

\section{Example: Lorentzian spectral line}

Example from spectroscopy.

Parameters: centre $x_0$, width $\Gamma$, amplitude $A$ and background $B$. Overall optimal design (compare to uniform on a sensible range), check effect of $N \rightarrow \infty$, optimize for $x_0$ or $\gamma$ (compare to uniform), do the same for one-sided (ion trapping). calculate optimal/sub-optimal for imprecise $\hat x_0$ or $\hat\gamma$. Check compared to normal distribution (model fit)???

\section{What have we learned?}

\ack Thanks for the dudes and to the National Science Council of Taiwan.

\section*{References}
\bibliographystyle{unsrt}
\bibliography{optimal}
\end{document}
